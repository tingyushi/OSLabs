\documentclass[12pt]{article}

\usepackage{graphicx}
\usepackage{paralist}
\usepackage{listings}
\usepackage{booktabs}
\usepackage{hyperref}
\usepackage{amsfonts}
\usepackage{amsmath}
\usepackage{color}
\usepackage{fancyhdr}
\usepackage{geometry}
\usepackage{subcaption}
\usepackage{mwe}

\geometry{margin = 0.75in}

\title{Memory Management Report}
\author{
Tingyu Shi(400253854)\\
Jiacheng Wu(400207981)}
\date{\today}
\begin {document}
\maketitle %%%%%%%%make title
\newpage
\section{Analysis}
\subsection{Three algorithms without Compaction}
For the sample output, please check section 2.1.\\
For the code, please check section 3.1.\\
he following are results of 10 trials.
\begin{table}[h!]
\centering
\begin{tabular}{|c|c|c|c|}
\hline
         & First Fit & Best Fit & Worst Fit \\ \hline
Trial 1  & 72        & 72       & 72        \\ \hline
Trial 2  & 132       & 132      & 132       \\ \hline
Trial 3  & 120       & 120      & 120       \\ \hline
Trial 4  & 232       & 232      & 232       \\ \hline
Trial 5  & 184       & 184      & 184       \\ \hline
Trial 6  & 176       & 176      & 176        \\ \hline
Trial 7  & 160       & 160      & 160       \\ \hline
Trial 8  & 96        & 96       & 96       \\ \hline
Trial 9  & 48        & 48       & 48       \\ \hline
Trial 10 & 148       & 148      & 148        \\ \hline
\end{tabular}
\caption{Hole Size(KB) Information after removing 10\% }
\end{table}

\noindent Average Hole size after removing 10\% is 136.8(KB) 

\begin{table}[h!]
\centering
\begin{tabular}{|c|c|c|c|}
\hline
         & First Fit & Best Fit & Worst Fit \\ \hline
Trial 1  & 8        & 8        & 8        \\ \hline
Trial 2  & 8        & 20       & 56       \\ \hline
Trial 3  & 20       & 20       & 20       \\ \hline
Trial 4  & 24       & 24       & 24       \\ \hline
Trial 5  & 24       & 0        & 44       \\ \hline
Trial 6  & 16       & 16       & 40        \\ \hline
Trial 7  & 36       & 36       & 36       \\ \hline
Trial 8  & 20       & 20       & 60       \\ \hline
Trial 9  & 28       & 28       & 28       \\ \hline
Trial 10 & 60       & 12       & 60        \\ \hline
\end{tabular}
\caption{Hole Size(KB) Information after refilling }
\end{table}
\noindent Average Hole size after refilling(First Fit) is 24.4(KB)\\
Average Hole size after refilling(Best Fit) is 18.4(KB)\\
Average Hole size after refilling(Worst Fit) is 37.6(KB)\\

\noindent Frist Fit memory utilization rate $$= \frac{136.8 - 24.4}{136.8} = 82.16\%$$\\
Best Fit memory utilization rate $$= \frac{136.8 - 18.4}{136.8} = 86.55\%$$\\
Worst Fit memory utilization rate $$= \frac{136.8 - 37.6}{136.8} = 72.51\%$$\\

\noindent Conclusion: When there is no compaction, $Best\ Fit > First\ Fit > Worst\ Fit$. The experiment result may be caused by the following factors:
\begin{itemize}
\item Best Fit tries to find the smallest hole which can fit the process. As
a result, the hold size left is the smallest.
\item The experiment is designed in a way that each process's size is a 
multiple of 4KB and the range is [4KB, 100KB]. This may lead to inaccurate 
experiment result. If we can use random process size, the result may be more
accurate.
\end{itemize}
\newpage
\subsection{Three algorithms with Compaction}
For the sample output, please check section 2.2.\\
For the code, please check section 3.2.\\
The following are results of 10 trials.
\begin{table}[h!]
\centering
\begin{tabular}{|c|c|c|c|}
\hline
         & First Fit & Best Fit & Worst Fit \\ \hline
Trial 1  & 56        & 56       & 56        \\ \hline
Trial 2  & 120       & 120      & 120       \\ \hline
Trial 3  & 168       & 168      & 168       \\ \hline
Trial 4  & 148       & 148      & 148       \\ \hline
Trial 5  & 100       & 100      & 100       \\ \hline
Trial 6  & 84        & 84       & 84        \\ \hline
Trial 7  & 184       & 184      & 184       \\ \hline
Trial 8  & 140       & 140      & 140       \\ \hline
Trial 9  & 144       & 144      & 144       \\ \hline
Trial 10 & 92        & 92       & 92        \\ \hline
\end{tabular}
\caption{Hole Size(KB) Information after removing 10\% and after compaction}
\end{table}

\begin{table}[h!]
\centering
\begin{tabular}{|c|c|c|c|}
\hline
         & First Fit & Best Fit & Worst Fit \\ \hline
Trial 1  & 0        & 0       & 0        \\ \hline
Trial 2  & 20       & 20      & 20       \\ \hline
Trial 3  & 12       & 12      & 12       \\ \hline
Trial 4  & 24       & 24      & 24       \\ \hline
Trial 5  & 4       & 4      & 4       \\ \hline
Trial 6  & 0        & 0       & 0        \\ \hline
Trial 7  & 8       & 8      & 8       \\ \hline
Trial 8  & 8       & 8      & 8       \\ \hline
Trial 9  & 12       & 12      & 12       \\ \hline
Trial 10 & 12        & 12       & 12        \\ \hline
\end{tabular}
\caption{Hole Size(KB) Information after refilling}
\end{table}
\noindent For each trial, after removing 10\% and compaction, there is only
one hole in the memory and the hole sizes are the same because the program
removes the same 10\% of the processes. Since there is only one hole in the 
memory, three algorithms perform in the same way. As as a result, in table 4,
three algorithms show the same hole sizes for each trial after refilling.\\
Conclusion: Three algorithms perform the same after compaction as there is
only one hole in the memory after compaction.
\newpage
\section{Sample Outputs}
\small
\subsection{Sample output of three algorithms without Compaction}
\lstinputlisting{../src/No_Compaction_Sample_Results}
\newpage
\subsection{Sample output of three algorithms with Compaction}
\lstinputlisting{../src/Compaction_Sample_Results}
\newpage
\scriptsize
\section{Code}
\subsection{Three algorithms without Compaction}
\lstinputlisting[language=C]{../src/Three_Fits.c}
\newpage
\subsection{Three algorithms with Compaction}
\lstinputlisting[language=C]{../src/Compaction_Three_Fits.c}
\end {document}